\documentclass[12pt]{article} 
\usepackage[utf8]{inputenc}
\usepackage[spanish,activeacute]{babel}
\usepackage{amsfonts}
\usepackage{lineno,hyperref}
\modulolinenumbers[5]
\usepackage{amsmath, amssymb , graphicx }
\usepackage{algorithm} 
\usepackage[noend]{algpseudocode}

\newenvironment{proof}[1][Proof.]{ \begin{trivlist}
\item[\hskip \labelsep {\bfseries #1}]}{\end{trivlist}}
\usepackage{tikz}
\usetikzlibrary{shapes.geometric, arrows}
\usepackage{booktabs}
\usepackage{isomath}
\usepackage{latexsym, amscd, amsfonts, enumerate, color}



\usepackage{xcolor}
\usepackage{listings}

% -----------------------------
% رنگ‌ها
% -----------------------------
\definecolor{mapleKeyword}{RGB}{0,0,180} % آبی برای کلیدواژه‌ها
\definecolor{mapleComment}{RGB}{0,128,0} % سبز برای کامنت
\definecolor{mapleNumber}{RGB}{180,0,0} % قرمز برای اعداد
\definecolor{mapleString}{RGB}{128,0,128} % بنفش برای رشته‌ها
\definecolor{mapleOperator}{RGB}{128,0,0} % 


\lstdefinelanguage{MapleCustom}{
morekeywords={restart,with,proc,end,return,local,for,from,to,do,if,then,else,fi,break},
sensitive=true,
morecomment=[l]{#},
morestring=[b]",
}

\lstset{
language=MapleCustom,
basicstyle=\ttfamily\small,
keywordstyle=\color{mapleKeyword}\bfseries,
commentstyle=\color{mapleComment}\itshape,
stringstyle=\color{mapleString},
numbers=left,
numberstyle=\tiny\color{gray},
stepnumber=1,
numbersep=8pt,
showstringspaces=false,
breaklines=true,
frame=single,
tabsize=4,
literate=
{:=}{{\color{mapleOperator}{:=}}}2
{+}{{\color{mapleOperator}{+}}}1
{-}{{\color{mapleOperator}{-}}}1
{*}{{\color{mapleOperator}{*}}}1
{/}{{\color{mapleOperator}{/}}}1
{=}{{\color{mapleOperator}{=}}}1
{<}{{\color{mapleOperator}{<}}}1
{>}{{\color{mapleOperator}{>}}}1
{<=}{{\color{mapleOperator}{<=}}}2
{>=}{{\color{mapleOperator}{>=}}}2
{<>}{{\color{mapleOperator}{<>}}}2
}

\newtheorem{class}{Class }
\newtheorem{theorem}{Theorem }
\newtheorem{remark}{Remark}
\newtheorem{example}{Example }
\newtheorem{corollary}{Corollary }
\newtheorem{definition}{Definition }
\newtheorem{hypothesis}{Hypothesis H1 }
\newtheorem{problem}{Problem }

\bibliographystyle{elsarticle-num}
%%%%%%%%%%%%%%%%%%%%%%%
\makeatletter
\setlength{\textwidth}{160mm}
\setlength{\textheight}{250mm}
\setlength{\topmargin}{-20mm}
\setlength\oddsidemargin {1\p@}
\setlength\evensidemargin {7\p@}
\setlength\marginparwidth{0\p@}
\setlength\headsep {0\p@}
\newdimen\bibindent
\setlength\bibindent{\parindent}
\setlength{\parskip}{\z@ \@plus \p@}
\setlength{\hfuzz}{2\p@}
\setlength{\arraycolsep}{1.5\p@}
\tolerance=500
\predisplaypenalty=0
\clubpenalty=10000
\widowpenalty=10000
\setlength\footnotesep{7.7\p@}
\newdimen\betweenumberspace % dimension for space between
\betweenumberspace=5\p@ % number and text of titles
\newdimen\headlineindent % dimension for space of
\headlineindent=2.5cc % number and gap of running
\setlength{\footskip}{15mm}


\makeatother


%%%%%%%%%%%%%%%%%%%%%%%
%% Elsevier bibliography styles
%%%%%%%%%%%%%%%%%%%%%%%
%% To change the style, put a % in front of the second line of the current style and
%% remove the % from the second line of the style you would like to use.
%%%%%%%%%%%%%%%%%%%%%%%

%% Numbered
%\bibliographystyle{model1-num-names}

%% Numbered without titles
%\bibliographystyle{model1a-num-names}

%% Harvard
%\bibliographystyle{model2-names.bst}\biboptions{authoryear}

%% Vancouver numbered
%\usepackage{numcompress}\bibliographystyle{model3-num-names}

%% Vancouver name/year
%\usepackage{numcompress}\bibliographystyle{model4-names}\biboptions{authoryear}

%% APA style
%\bibliographystyle{model5-names}\biboptions{authoryear}

%% AMA style
%\usepackage{numcompress}\bibliographystyle{model6-num-names}

%% `Elsevier LaTeX' style
\bibliographystyle{elsarticle-num}




\begin{document}
\subsection*{Response to Reviewer 1:}
The paper introduces hybrid radial kernels to improve the accuracy-stability balance in RBF-based collocation methods and optimizes kernel parameters using a modified PSO algorithm. The idea is interesting and relevant, and the numerical results indicate potential advantages over standard kernels. The convergence analysis is well presented and helps strengthen the theoretical foundation. The paper is scientifically rigorous, and the proposed model and methodology are both effective and compelling, making it suitable for publication. Therefore, I recommend minor revision, with the following suggestions aimed at further strengthening the manuscript.
\\ \\  \vspace*{0.7cm} \\
\begin{Large}
\textbf{List of minor revisions}
\end{Large}
 \vspace*{0.7cm} \\
{\color{red}Comment: } The paper compares HRKs mainly with HS-SVD and Haar wavelet methods. Since the claims target improvements over classical RBF approaches, comparison with RBF-QR, standard GA/MQ collocation, or local RBF methods is recommended.  
\vspace*{0.5cm} \\
{\color{red}Response:} {\color{blue} We thank the reviewer for this valuable comment. The main objective of the paper is to develop a hybrid kernel-based approach for the numerical solution of Volterra integral equations. Accordingly, in the numerical section we compared our method with HS-SVD and Haar wavelet schemes, which are well-established and widely used techniques for such equations. In addition, several of our figures include comparisons with classical RBF kernels, such as GA and MQ, to illustrate the performance of the proposed hybrid method relative to standard RBF approaches.

Since the focus of the study is on applying the proposed hybrid RBFs strategy to Volterra integral problems—not on providing an extensive survey of all RBF stabilization techniques—we believe that the comparisons with HS-SVD  adequately demonstrate the effectiveness and advantages of the proposed method within the intended scope of the paper. Nevertheless, we appreciate the reviewer’s suggestion and acknowledge that broader comparisons could be of interest for future studies.  }
 \vspace*{0.5cm} \\
{\color{red}Comment: } The optimization step may be computationally costly; however, the paper does not provide sufficient attention to the timing or convergence behavior of PSO. Reporting this information would be useful for assessing feasibility, if possible.
 \vspace*{0.5cm} \\
{\color{red}Response:} {\color{blue} Thank you for pointing this out.  In response, we would like to point out that in Examples 1 and 2, the convergence behavior of Gbest  for the parameters $\rho$ and $\varepsilon$ has been presented. In addition, the full implementation of the proposed PSO algorithm has been made publicly available via an external link, which is now provided in the revised manuscript.  This information has now been included in the revised manuscript to provide better insight into the optimization process.}
\vspace*{0.5cm} \\
{\color{purple} For more details, please see Section 6 (Numerical Results).}
 \vspace*{0.5cm} \\
{\color{red}Comment: } No theoretical or empirical basis is given for choosing the parameter bounds used in PSO. A sensitivity analysis or an explanation should be added.
\vspace*{0.5cm} \\
{\color{red}Response:} {\color{blue} We thank the reviewer for this comment. The bounds for $\varepsilon$ and $\rho$ were chosen based on theoretical and empirical considerations. Theoretically, very small $\varepsilon$ causes ill-conditioning, while very large $\varepsilon$ reduces accuracy. 
On the other hand, the parameter $\rho$ should be chosen sufficiently large to enhance numerical stability, while remaining small enough to ensure rapid convergence. Empirically, tests showed that the optimal solutions consistently lie within the chosen ranges $\varepsilon \in  [0,10]$ and $\rho \in  [0,1]$ and enlarging the bounds did not improve accuracy. A brief sensitivity discussion has been added to the revised manuscript to justify these PSO search domains.}
 \vspace*{0.5cm} \\{\color{
purple}(Please see page 11
in the revised manuscript.)}
\vspace*{0.5cm} \\
{\color{red}Comment: } The boundedness of $Q_n$ and $(I-W_1)$ are assumed without citing precise conditions. The authors should specify the required assumptions, such as quasi-uniform nodes or Lipschitz conditions on the kernel.
\vspace*{0.5cm} \\
{\color{red}Response:} {\color{blue} We are very grateful to your comments on the manuscript. In the revised manuscript, we stated the conditions under which these operators are bounded, such as assuming quasi-uniform nodes and Lipschitz condition of the kernel. } \vspace*{0.5cm} \\{\color{
purple}(Please see pages 6 and 9
in the revised manuscript.)}\\
 \vspace*{0.5cm}
\\ 
{\color{red}Comment: } The Introduction should be expanded to include more details about the applications of Volterra integral equations, as these equations have a wide range of applications in various fields. 
\vspace*{0.5cm} \\
{\color{red}Response:} {\color{blue} Thank you for  your  valuable  comments.  We have added this content in the revised version as follow: }\vspace*{0.35cm} \\
{\color{green} “VIEs have many applications in various areas such as:  
\begin{itemize}
\item[$\bullet$] The following nonlinear Volterra integral equation (VIE) arises in the analysis of the neural networks with a post-inhibitory rebound [2], 
\[
u(x)=1+ \int_{0}^{x} (x-t)^3  (4-x+t) e^{t-x} \frac{u^4(t)}{1+2u^2(t)+2u^4(t)} dt, \quad x \in [0,10].
\]

\item[$\bullet$] The following VIE arises in the analysis of the reflexion of sound pulses [3], 
\[
u(x)=f(x)- \int_{0}^{x}  \frac{2}{(x-t+2)^2}g(u(t)) dt, \quad x \in [0,40].
\]


\item[$\bullet$] Some nonlinear VIEs arise as a reformulation of initial value problems which occur in many
problems of mathematical physics. For example, consider the Lane-Emden type equations arise in the analysis of
the gravitational potential of the degenerate white-dwarf stars, the isothermal gas sphere, the static stellar models
in Newtonian gravity and other problems in the mathematical physics [3], which are defined as follows: 
\[
u^{''}+\frac{2}{s}u^{'}+g(s)f(u)=h(s), \quad 0<s< \infty,
\]
subject to 
\[
u(0)=a, \quad u^{'}(0)=0, 
\]
where $a$ is a constant and $f,g$ and $h$ are determined functions. The given Lane-Emden problem can be transformed
into the following VIE [3,4]
\[
u(x)=a+\int_{0}^{x} \Bigl ( \frac{t^2}{x}-t \Bigr ) (g(u) f(u(t)-h(t))) dt, \quad x \in (0,\infty).
\] 
\end{itemize} } \vspace*{0.5cm} \\
{\color{
purple}(Please see pages 1 and 2
in the revised manuscript.)}
\vspace*{0.5cm} \\
{\color{red}Comment: } The authors should provide information about the computational environment used in the study. Specific details about the hardware and software configurations should be included for a more complete understanding of the numerical experiments. 
\vspace*{0.5cm} \\
{\color{red}Response:} {\color{blue} Thank you very much for your suggestion.  We have added this content in the revised version as follow:}\vspace*{0.35cm} \\{\color{green}“All calculations and plots have been done by “Maple 18” software and run on a Laptop with 1.70GHz of Core i5-4210U CPU and 6 GB of RAM.” } \vspace*{0.5cm}
\\ {\color{
purple}(Please see page 13
in the revised manuscript.)}\\
 \vspace*{0.5cm}
\\ 
{\color{red}Comment: } The authors may wish to reference relevant meshless methods such as the meshless local Galerkin method for solving Volterra integral equations derived from nonlinear fractional differential equations, nonlinear Volterra integral equations of the second kind using radial basis functions, and the meshless local discrete Galerkin (MLDG) scheme for numerically solving two-dimensional nonlinear Volterra integral equations.
\vspace*{0.5cm} \\
{\color{red}Response:} {\color{blue} Thank you for your valuable comments. Now, in the revised manuscript, we have
added all relevant references.}
  \vspace*{0.5cm} \\
{\color{red}Comment: } To improve the interpretation of the results shown in the figures, the authors could provide additional commentary. Furthermore, an analysis of the effect of key parameters would help in gaining a better understanding of the observed patterns and outcomes.
\vspace*{0.5cm} \\
{\color{red}Response:} {\color{blue} Thank you for your comment. We have added further explanations to the figures and included an analysis of key parameters to clarify the observed patterns. }
 \vspace*{0.5cm} \\
{\color{red}Comment: } While the manuscript is well written, minor improvements in language and presentation could enhance overall readability. 
\vspace*{0.5cm} \\
{\color{red}Response:} {\color{blue} Thank you for your valuable feedback. We have carefully reviewed the comments regarding language and presentation and have implemented the necessary revisions to improve the clarity and readability of the manuscript. }
 \vspace*{0.5cm} \\



\hspace*{-0.6cm}{\color{violet}The authors would like to thank the reviewer for the attention and the opportunity they gave us to
correct the issues, and also hope these answers be proper. } {\color{red}(All changes are shown in red color in
therevised manuscript.)}\\ \\
{\color{violet}With best regards and most sincerely,\\ \\
Corresponding author (on behalf of the authors)}


\end{document}


